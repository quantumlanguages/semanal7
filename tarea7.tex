% tipo
\documentclass{article}

% formato
\usepackage[letterpaper, margin = 1.5cm]{geometry}

% matematicas
\usepackage{amsmath}

% encabezado

\title{
    Lenguajes de Programacón 2020-1\\
    Facultad de Ciencias UNAM\\
    Ejercicio Semanal 7
}

\author{
    Sandra del Mar Soto Corderi\\
    Edgar Quiroz Castañeda
}

\date{
    3 de octubre del 2019
}

\begin{document}
    \maketitle

    \begin{enumerate}
        \item {
            Define la función \texttt{fibonacci} en el lenguaje \texttt{PCF}
            utilizando el operador \texttt{fix}.

            \begin{align*}
                &\texttt{
                    fibonacci = 
                    fun fib (n:Nat):Nat =>           
                    if (or iszero(n) iszero(n-1)) then 1 else fib(n-1) + fib(n-2)
                }\\
                &\texttt{
                    = fix fib:Nat -> Nat
                    => (fun(n:Nat) 
                    => if (or iszero(n) iszero(n-1)) then 1 else fib(n-1) + fib(n-2))
                }
            \end{align*}
                
        }
        \item {
            Evalúa la expresión \texttt{fibonacci 4} con la definición del
            inciso anterior.

            Tomemos

            \[
                \texttt{e := if (or iszero(n) iszero(n-1)) then 1 else fib(n-1) + fib(n-2))}
            \]

            \begin{align*}
                &\texttt{
                    fibonacci 4 = 
                    (fun fib (n:Nat):Nat => e) 4
                } \\
                &=\texttt{
                    (fix fib:Nat -> Nat
                    => (fun(n:Nat) => e)
                    ) 4
                } \\
                &\rightarrow 
                \texttt{
                    (fun(n:Nat) => (or iszero(4) iszero(4-1))
                    then 1
                    else fibonacci(n-1) + fibonacci(n-1)) 4
                } \\
                &\rightarrow
                \texttt{
                    if (or iszero(4) iszero(4-1))
                    then 1 
                    else fibonacci(4-1) + fibonacci(4-2))
                }
            \end{align*}
        }
    \end{enumerate}
\end{document}